\documentclass[a4paper,12pt]{extreport}
\usepackage[utf8]{inputenc}
\usepackage[T2A]{fontenc}
\usepackage[russian]{babel}
\usepackage{amsmath}
\usepackage{amssymb}
\usepackage{graphicx}
\usepackage{array}
\usepackage{multirow}
\usepackage{booktabs}
\usepackage{geometry}
\usepackage{float}
\usepackage{ulem}
\usepackage{tabularx}
\usepackage{booktabs}
\usepackage{siunitx}
\usepackage{indentfirst}
\usepackage{placeins}
\usepackage{lipsum}
\usepackage{longtable}
\usepackage{csquotes}
\usepackage[style=gost-numeric,sorting=none]{biblatex}
\usepackage{enumitem}
\usepackage{caption}
\captionsetup[table]{position=top, singlelinecheck=false, justification=raggedleft}



\addbibresource{references.bib}
\usepackage{titlesec}
\titleformat{\chapter}[display]
    {\normalfont\huge\bfseries}
    {\chaptertitlename\ \thechapter}
    {20pt}
    {\Huge}
\usepackage{booktabs}
\usepackage{makecell}
\graphicspath{{Image/}}

\geometry{
    a4paper,
    total={170mm,257mm},
    left=20mm,
    top=20mm,
}

\newcommand{\specialcell}[2][c]{%
    \begin{tabular}[#1]{@{}c@{}}#2\end{tabular}%
}

\usepackage{etoolbox}
\patchcmd{\thebibliography}
  {\chapter*{\bibname}}
  {\section*{\centering СПИСОК ЛИТЕРАТУРЫ}}
  {}{}
\patchcmd{\thebibliography}
  {\addcontentsline{toc}{chapter}{\bibname}}
  {\addcontentsline{toc}{section}{СПИСОК ЛИТЕРАТУРЫ}}
  {}{}


\usepackage{setspace}
\onehalfspacing


\begin{document}





\begin{center}
    \begin{tabular}{cc}

        \specialcell{\includegraphics[width=0.8in,height=0.9in]{images/BMSTU.jpg}} & 
        \specialcell{
            \textbf{Министерство науки и высшего образования Российской Федерации} \\
            \textbf{Федеральное государственное автономное образовательное учреждение} \\
            \textbf{высшего образования} \\
            \textbf{«Московский государственный технический университет} \\
            \textbf{имени Н. Э. Баумана} \\
            \textbf{(национальный исследовательский университет)»} \\
            \textbf{(МГТУ им. Н. Э. Баумана)}
        } \\
   
    \end{tabular}
\end{center}

\vspace{1cm}

\begin{center}
    \textbf{ФАКУЛЬТЕТ \uline{«СПЕЦИАЛЬНОЕ МАШИНОСТРОЕНИЕ»}}
    
    \textbf{КАФЕДРА \uline{«РАКЕТНЫЕ И ИМПУЛЬСНЫЕ СИСТЕМЫ» (СМ-6)}}
    
    \vspace{1cm}
    
    \textbf{ДОМАШНЕЕ ЗАДАНИЕ}
    
    \vspace{1cm}
    
    ПО ДИСЦИПЛИНЕ:
    
    \vspace{0.5cm}
    
    \begin{tabular}{c}
       
        \specialcell{Проектирование ракетного оружия} \\
        \hline
    \end{tabular}
    
    \vspace{1cm}
    
    НА ТЕМУ:
    
    \vspace{0.5cm}
    
    \begin{tabular}{c}
       
        \specialcell{Массовый анализ AIM-120 AMRAAM} \\
        \hline
    \end{tabular}
    
    \vspace{1cm}

    
    \vspace{1cm}
    

    \begin{tabular}{llp{3cm}l}
       
        Выполнил: студент группы СМ6-62 & 
        (подпись, дата) & 
         & 
        Ерофеев М.В.\\
        \hline
    \end{tabular}
    
    \vspace{1cm}
    
    \begin{tabular}{llp{3cm}l}
    
        Проверил &  (подпись, дата) & & Лаптева Л.А.\\
        \hline
    \end{tabular}
    
    \vspace{6cm}
    
    Москва, 2025 г.
\end{center}
\tableofcontents

\newpage
\chapter*{Принятые сокращения}
\begin{description}[font=\normalfont\itshape] % обычный шрифт + курсив
\item[WDU] — Weapons Detonation Unit (блок подрыва боевой части)
\item[WGU] — Weapons Guidance Unit (блок наведения вооружения)
\item[WPU] — Weapons Propulsion Unit (двигательный отсек вооружения)
\item[ICPU] — Integrated Control and Power Unit (встроенный блок управления и питания)
\item[РДТТ] — ракетный двигатель твердого топлива
\item[БЧ] — боевая часть
\end{description}


\newpage
\chapter{Краткие сведения о прототипе}
\section{Обзор прототипа}
Ракета AIM-120A AMRAAM (Advanced Medium-Range Air-to-Air Missile –усовершенствованная ракета класса «воздух-воздух»  средней дальности) выполнена по нормальной аэродинамической схеме с «Х» – образным расположением консолей крыла и рулей.

\begin{figure}[h!]
\centering
\includegraphics[width=0.7\textheight]{images/1.jpg}
\caption{Ракета AIM-120 AMRAAM}
\label{AIM-120}
\end{figure}

\newpage
\section{Внешнее описание}
Ракета цилиндрическая, длинная, со стреловидным обтекателем.
Носовая часть имеет длину 18.5 дюймов и окрашена в белый цвет. Далее расположена секция батарей серого цвета длиной 17.5 дюймов. Имеется желтая и чёрная полоса с надписью «Осторожно — Используйте защитный чехол для обтекателя».
Следом идет неокрашенная серая секция управления (WCU) длиной 18.75 дюймов. За ней расположена секция БЧ (WDU), длиной 9.5 дюймов, темно-серого цвета.
Эта секция снизу переходит в более светло-серую секцию РДТТ (WPU) длиной 74.75 дюймов. На ней в верхней части расположена черная и синяя полосы, обозначающие, что это учебный снаряд.
За секцией РДТТ находится секция управления рулями длиной 14.75 дюймов. Рули длинные, частично треугольной формы с прямым краем сверху. На них наклеены красно-белые полосы и нанесены номера.
Передние крылья также имеют наклейки и номера. Они алюминиевые, треугольной формы. На ракете присутствуют ушки для крепления к пилону.


\begin{figure}[h!]
\centering
\includegraphics[width=0.55\textheight]{images/2.jpg}
\caption{Ракета AIM-120 AMRAAM}
\label{AIM-120}
\end{figure}

\begin{figure}[h!]
\centering
\includegraphics[width=0.6\textheight]{images/3.jpg}
\caption{Размеры отсеков в миллиметрах}
\label{AIM-120}
\end{figure}

\newpage
\section{Внутрення компоновка}
На рисунке 1.4 представлена внутрення компоновка AIM-120, перевод названий модулей дан ниже.

\begin{figure}[h!]
\centering
\includegraphics[width=0.7\textheight]{images/4.jpg}
\caption{Компоновка ракеты AIM-120}
\label{AIM-120}
\end{figure}

\begin{description}[font=\normalfont\itshape] % обычный шрифт + курсив
\item[Antenna] — антенна головки самонаведения
\item[(Thermal) batteries] — пиротехнические баратареи, часть ICPU
\item[Transmitter] — передатчик, излучатель
\item[Electronics] — электроника 
\item[Inertial Reference Unit (IRA)] — инерциальная система наведения 
\item[Target Detecting Device (TDD)] — устройство обнаружения цели
\item[Armament Section] — боевая часть 
\item[Rocket Motor] — РДТТ
\item[Contol Actuator] — рулевая машинка
\item[Data Link] — канал передачи данных 
\end{description}

Ниже на рисунке 1.5 представлено разбиение компоновки ракеты на 4 отсека в соответсвии с требованиями ДЗ.

\begin{figure}[H]
\centering
\includegraphics[width=0.65\textheight]{images/5.jpg}
\caption{Схема разбиения компоновки ракеты}
\label{AIM-120}
\end{figure}

\chapter{Массовый анализ}
\section{Расчёт масс отсеков из размеров ракеты}
Расчёт основан на имеющихся данных о массе ракеты, её боевой части из открытых источников. Масса РДТТ бралась из ДЗ по ПрРО предыдущего семестра. Масса остальных отсеков будет найдена
с помощью установленных зависимостей из [1].

Общая масса ракеты — 161.5 кг, масса БЧ — 22 кг. Масса РДТТ — 49.37 кг.

\begin{tabular}{ll}
    \textbf{Дальность}: & 35 морских миль (64.82 км); \\
    \textbf{Скорость}: & 4 Маха; \\
    \textbf{Максимальная длина}: & 12.19 футов (3.717 м); \\
    \textbf{Калибр}: & 0.58 футов (0.178 м).
\end{tabular}

Первый шаг заключается в расчете общего объема ракеты на основе
указанных выше длины и калибра по следующей формуле

\[V = \frac{\pi D^2 L}{4} = \frac{\pi (0.58)^2 \cdot 12.19}{4} = \SI{3.22}{ft^3}\]

Для того что
бы получить оценочное значение массы, выбирается уравнение 4 из анализа общей массы УРВВ:
\[W = 142.2\cdot(V)^{0.74},\]
\[W = 142.2 \cdot (3.22)^{0.74} = 337.84\text{ фунтов}(153.24\text{ кг})\] 

Данное оценочное значение может быть проверено с помощью уравнения 17, разработанного для ракет средней дальности:

\[ W = 177.5\cdot(V)^{0.73} \]

\[ W = 177.5\cdot(3.38)^{0.73} = 416.93 \text{ фунтов}(189.11\text{ кг}) \]

 Поскольку полученные значения отличаются, проводится сравнение со
ответствия для каждого из них. Выбирается уравнение 4, по причине более высокого значения R — квадрат. Таким образом, значение массы при
 начальной оценке равно 350.18 фунтов (158.83 кг). Следовательно, при из
вестных массе и объеме общая плотность изделия может быть рассчитана
 с помощью уравнений:30,


\[ \textit{DENS} = \frac{\textit{W}}{\textit{V}} \]

\[ \textit{DENS} = 104.91 \, \frac{\text{фунтов}}{\text{фут}^3} . \]

Затем вводятся уравнения, разработанные для масс отсеков с параметрами, которые были выведены и оценены. Во-первых, масса отсека ДУ
может быть оценена с помощью уравнения 77:

\[ \textit{PWt} = -284.9 + 633.6(D) - 0.105(W) + 0.949(\textit{DENS}); \]
\[ \textit{PWt} = -284.9 + 633.6(0.58) - 0.105(337.84) + 0.949(104.91) = 146.67 \text{ фунтов}(66.52 \text{ кг})\]


Данное значение проверяется уравнением 82:

\[ \textit{PWt} = 1548.0 - 43.7(L) - 1253.9(D) + 1.4(W) - 6.0(\textit{DENS}); \]
\[ \textit{PWt} = 1548.0 - 43.7(12.19) - 1253.9(0.58) + 1.4(337.84) - 6.0(104.91) = 131.551 \text{ фунтов}(59.67 \text{ кг}) \]

Эти уравнения дали большое расхождение. Так как уравнение
82 имеет лучшее соответствие значению массы РДТТ из ДЗ, для определения
массы отсека ДУ будет использоваться значение 131.551 фунтов (59.67).

 Масса и размер отсека наведения и управления будут оценены анало
гичным образом: оценка массы отсека будет получена из уравнения 85:

\[ GCWt = 117.6(D) + 1.6(R) - 0.14(\textit{DENS}); \]
\[ GCWt = 117.6(0.58) + 1.6(35) - 0.14(104.91) = 109.52 \text{ фунтов}( 49.67\text{ кг}) \]


Теперь определим массу и размеры отсека боевой части. Для оценки массы используем уравнение 93:

\[ WHWt = 0.1(DENS) - 0.2(R) + 0.2(W) - 2.4(L); \]
\[ WHWt = 0.1(104.91) - 0.2(35) + 0.2(337.84) - 2.4(12.81) = 40.315 \text{ фунтов}( 18.28\text{ кг}) \]


Масса рулевого отсека будет рассчитана из общей массы ракеты:

\[
ROW = Wt - GCWt - PWt - WHWt = 337.84  - 109.52 - 131.551 - 40.315 = 56.454 \, \text{ фунтов}( 25.60\text{ кг})
\]

Итого: 
\captionsetup[table]{position=top, singlelinecheck=false, justification=raggedleft}
\begin{table}[H]
\caption{Результаты анализа}
\label{tab:three_columns}
\centering
\begin{tabular}{|c|c|c|}
\hline
\textbf{Отсек} & \textbf{Имеющиеся данные, кг} & \textbf{Регрессионный анализ, кг} \\
\hline
БЧ & 22 & 18.28 \\
\hline
РДТТ & 49.37 & 59.67 \\
\hline
Блок управления & - & 49.67 \\
\hline
Рулевой отсек & - & 25.60 \\
\hline
Общая масса ракеты & 161.5 & 153.22 \\
\hline
\end{tabular}
\end{table}

Примем реальное значение боевой части, массу РДТТ  возьмем из ДЗ, а массу блока управления и рулевого отсека возьмём из регрессонного анализа. Тогда:

\captionsetup[table]{position=top, singlelinecheck=false, justification=raggedleft}

\begin{table}[H]
\caption{Принятые массы}
\centering
\begin{tabular}{|c|c|}
\hline
\textbf{Отсек} & \textbf{Масса отсека, кг} \\
\hline
БЧ & 22 \\
\hline
РДТТ & 49.37\\
\hline
Блок управления & 49.67\\
\hline
Рулевой отсек & 25.60 \\
\hline
Общая масса ракеты & 146.64 \\
\hline
\end{tabular}
\label{tab:three_columns}
\end{table}

\section{Расчёт плотностей отсеков}

Расчёт будет производиться по формуле:

\[ \rho = \frac{m}{V}, \quad \left[\frac{\text{кг}}{\text{м}^3}\right] \]

Но сначала необходимо высчитать объемы отсеков по формуле: 

\[V = \frac{\pi D^2 L}{4} \]

1) Объём боевой части:

\[V_{\text{БЧ}} = \frac{\pi (0.178)^2 \cdot 0.241}{4} = 0.0059{\text{ м}^3} \]

2) Объём РДТТ:

\[V_{\text{РДТТ}} = \frac{\pi (0.178)^2 \cdot 1.716}{4} = 0.042{\text{ м}^3} \]

3) Объём блока управления:

\[V_{\text{БУ}} = \frac{\pi (0.178)^2 \cdot 0.921}{4} = 0.0229{\text{ м}^3} \]

4) Объём рулевого отсека(блока рулей):

\[V_{\text{БР}} = \frac{\pi (0.178)^2 \cdot 0.375}{4} = 0.0093{\text{ м}^3} \]

Плотности:

1) Плотность боевой части:

\[ \rho_{\text{БЧ}} = \frac{22}{0.0059} = 3728.81 \quad \left[\frac{\text{кг}}{\text{м}^3}\right]\]

2) Плотность РДТТ:

\[ \rho_{\text{РДТТ}} = \frac{49.37}{0.042} = 1175.47 \quad \left[\frac{\text{кг}}{\text{м}^3}\right]\]

3) Плотность блока управления:

\[ \rho_{\text{БУ}} = \frac{49.67}{0.0299} = 1661.2 \quad \left[\frac{\text{кг}}{\text{м}^3}\right]\]

4) Плотность рулевого отсека (блока рулей):

\[ \rho_{\text{БР}} = \frac{25.60}{0.0093} = 2752.68 \quad \left[\frac{\text{кг}}{\text{м}^3}\right]\]


\chapter{Расчёт центра масс}

\section{Построение 3D модели}

Для построения 3D модели используем САПР Компас 3D. Каждый отсек ракеты мо-
делируем отдельно и указываем его плотность и массу. Создаём сборку (рис. 3.1) с полной
массой топлива и без топлива и смотрим на свойства модели.

\begin{figure}[h]
\centering
\includegraphics[width=0.65\textheight]{images/6.jpg}
\caption{Сборка ракеты в САПР Компас 3D}
\label{AIM-120}
\end{figure}

\section{Вычисление центра масс}

Вычисление проводилось с помощью САПР Компас 3D. Результаты приведены в таблице 3.1.


\begin{table}[h]
\caption{Координаты центра масс до и после работы двигателя}
\label{tab:four_columns}
\centering
\begin{tabular}{|p{0.2\textwidth}|p{0.2\textwidth}|}
\hline
\textbf{Состояние топливного заряда} & \textbf{x, мм} \\
\hline
Полная загрузка топлива &  2103.12 \\
\hline
Топливо отсутствует & 1914.05 \\
\hline
\end{tabular}
\end{table}

\newpage
Зависимость положения центра тяжести от времени x(t) будет выглядеть так: 

\[x = -11.816\cdot t + 2103.12 \]

Стоить отметить, что время горения шашки твердового топлива составляет 16 секунд (данные из ДЗ по ПрРО за прошлый семестр).

Разбежка центра тяжести составила:

\[ \varDelta_{\text{цт}} = \varDelta_{\text{цт2}} - \varDelta_{\text{цт1}} = 2103.12 - 1914.05 = 189.07{\text{ мм}} \]

Относительная разбежка центра тяжести составила:

\[
\frac{|\varDelta_{\text{цт}}|}{L_{\varSigma}} \cdot 100\% = \frac{|189.07{\text{ мм}}|}{3717{\text{ мм}}} \cdot 100\% = 5.08\%
\]

\begin{figure}[h]
\centering
\includegraphics[width=0.65\textheight]{images/7.jpg}
\caption{Эскиз ракеты с координатами центров масс}
\label{AIM-120}
\end{figure}

\chapter{Аеродинамические характеристики}

\section{Некоторые пояснения}

В зависимости от связанной или скоростной системы координат расчёт $c_y$ производится по формулам \ref{eq:cy_formula1} и \ref{eq:cy_formula2} соответственно.

\begin{equation}
\label{eq:cy_formula1}
c_y = c_{y_a0} + c_{y_a}^\alpha \cdot \alpha + c_{y_a}^{\delta \mathrm{I}} \cdot \delta \mathrm{I} + c_{y_a}^{\delta \mathrm{II}} \cdot \delta \mathrm{II},
\end{equation}

\begin{equation}
\label{eq:cy_formula2}
c_y = c_{y_a10} + c_{y_a1}^\alpha \cdot \alpha + c_{y_a1}^{\delta \mathrm{I}} \cdot \delta \mathrm{I} + c_{y_a1}^{\delta \mathrm{II}} \cdot \delta \mathrm{II},
\end{equation}

где $\alpha$ -- угол атаки (в радианах), $c_{y_a0}$ и $c_{y_a10}$ -- значения $c_y$ и $c_{y1}$ при $\alpha = \delta_\mathrm{I} =  \delta_\mathrm{II} = 0$, $c_{y_a}^\alpha, c_{y_a}^{\delta \mathrm{I}},c_{y_a}^{\delta \mathrm{II}},c_{y_a1}^\alpha,c_{y_a1}^{\delta \mathrm{I}},c_{y_a1}^{\delta \mathrm{II}}$ -- частные производные коэффицентов $c_y$ или
$c_{y1}$ по углам $\alpha, \delta_\mathrm{I}$ и $\delta_\mathrm{II} $, взятые $\alpha = \delta_\mathrm{I} =  \delta_\mathrm{II} = 0$.

Данное соотношение справедливо для линейного диапазона, т.е. $sin(\alpha)\approx \alpha $.


Некоторые вводные данные представлены в таблице \ref{tab:geometry}.

\begin{table}[h]
\centering
\caption{Геометрические параметры летательного аппарата}
\label{tab:geometry}
\begin{tabular}{|l|c|c|}
\hline
\textbf{Параметр} & \textbf{Значение} & \textbf{Единица измерения} \\ 
\hline
Диаметр миделя $D$ & 0.178 & м \\
\hline
Диаметр в области передних консолей $D_I$ & 0.178 & м \\
\hline
Диаметр в области задних консолей $D_{II}$ & 0.178 & м \\
\hline
Относительный диаметр корпуса $\bar{D}$ & 0.6952 & -- \\
\hline
Длина фюзеляжа $l_f$ & 3.906 & м \\
\hline
Длина носовой части $l_{nos}$ & 0.47 & м \\
\hline
Длина кормовой части $l_{korm}$ & 0.375 & м \\
\hline
Площадь миделя $S_m$ & 0.02488 & м\textsuperscript{2} \\
\hline
Размах крыла $l_I$ & 0.484 & м \\
\hline
Размах руля $l_{II}$ & 0.5808 & м \\
\hline
\end{tabular}
\end{table}

Относительный диаметр корпуса $\bar{D}$:

\[ 
\bar{D} = \frac{D}{l_f} = \frac{0.178}{3.906} = 0.6952 
 \]

Площадь миделя $S_m$:

\[ 
S_m = \frac{\pi D^2}{4} = \frac{\pi 0.178^2}{4} = 0.02488{\text{ м}^2}
 \]
          
Коэффициенты интерференции рассчитываются по следующим формулам:

\[ 
k_{\alpha \alpha_{\text{теор}}} = (1 + 0.41 \cdot \bar{D})^2 = (1 + 0.41 \cdot 0.6952)^2 = 1.6513
\]

\[ 
K_{\alpha \alpha_{\text{теор}}} = (1 + \bar{D})^2 = (1 + 0.6952)^2 = 2.8737
 \]

\section{АДХ фезюляжа}

При малых углах атаки, на участке линейной зависимости, коэффициент подъемной силы фюзеляжа можно представить в виде:

\[
C_{ya\text{ф}} = C_{ya\text{ф}}^{\alpha} \cdot \alpha
\]

Коэффициент нормальной силы фюзеляжа при безотрывном обтекании определяется:

\[
C_{y\text{ф}} = C_{y\text{ф}}^{\alpha} \cdot \sin \alpha \cos \alpha
\]

Для малых углов атаки можно считать что

\[
C_{ya\text{ф}} = C_{y\text{ф}} \cos \alpha = C_{y\text{ф}}^{\alpha} \cdot \sin \alpha \cos^{2} \alpha,
\]

где можно принять \(\sin \alpha \approx \alpha\), \(\cos \alpha \approx 1\).

Тогда

\[
C_{ya\text{ф}}^{\alpha} \cdot \alpha = C_{y\text{ф}}^{\alpha} \cdot \alpha \quad \text{и} \quad C_{ya\text{ф}}^{\alpha} = C_{y\text{ф}}^{\alpha}
\]

Производная \( C_{y\text{ф}}^{\alpha} \) зависит от формы фюзеляжа и задается для эквивалентного тела вращения как:

\[
C_{y\text{ф}}^{\alpha} = C_{y\text{нос+цил}}^{\alpha} + C_{y\text{корм}}^{\alpha}
\]

где \( C_{y\text{нос+цил}}^{\alpha} \) -- производная \( C_{y\text{ф}}^{\alpha} \) носовой части фюзеляжа с учетом интерференции с цилиндрической частью;

\( C_{y\text{корм}}^{\alpha} \) -- производная \( C_{y\text{ф}}^{\alpha} \) кормовой части фюзеляжа.

\begin{figure}[H]
\centering
\includegraphics[width=0.7\textheight]{images/C_y_iz.f.png}
\caption{Схема разбиения компоновки ракеты}
\label{C_y_iz.f}
\end{figure}

\begin{figure}[H]
\centering
\includegraphics[width=0.7\textheight]{images/C_y_iz.f_V2.png}
\caption{Схема разбиения компоновки ракеты}
\label{C_y_iz.f_V2}
\end{figure}

% \subsection*{Параметры для функции \texttt{get\_c\_y\_alpha\_IsP}}

% \begin{align*}
% \lambda & \quad \text{удлинение несущей поверхности} \\
% \chi_{0.5} & = 0.5 \; \text{рад} \quad \text{угол стреловидности по линии середин хорд} \\
% \bar{c} & = 0.042 \quad \text{относительная толщина профиля} \\
% \zeta & = 1 \quad \text{обратное сужение несущей поверхности}
% \end{align*}

% \subsection*{Параметры для функции \texttt{get\_c\_y\_alpha\_NosCil\_Con}}

% \begin{align*}
% \lambda_{\text{нос}} & = \frac{l_{\text{нос}}}{D} = \frac{0.47}{0.178} \quad \text{удлинение носовой части} \\
% \lambda_{\text{цил}} & = \frac{l_f}{D} = \frac{3.906}{0.178} \quad \text{удлинение цилиндрической части}
% \end{align*}

% \subsection*{Геометрические параметры крыльевых поверхностей}

% \begin{align*}
% b_{\text{кр}} & = 0.1 \; \text{м} \quad \text{хорда крыла} \\
% b_{\text{оп}} & = 0.025 \; \text{м} \quad \text{хорда оперения} \\
% l_{\text{размах}} & = 0.144 \; \text{м} \quad \text{размах крыла} \\
% l_{\text{размах оп}} & = 0.066 \; \text{м} \quad \text{размах оперения} \\
% S_{\text{конс}} & = 2 \times 0.036 \times 0.1 = 0.0072 \; \text{м}^2 \quad \text{площадь консолей крыла} \\
% S_{\text{оп}} & = 2 \times 0.033 \times 0.025 = 0.00165 \; \text{м}^2 \quad \text{площадь оперения}
% \end{align*}

% \subsection*{Расчётные параметры крыльевых поверхностей}

% \begin{align*}
% l_{\text{размах конс}} & = l_{\text{размах}} - D = 0.144 - 0.178 \; \text{м} \\
% \lambda & = \frac{l_{\text{размах конс}}^2}{S_{\text{конс}}} \quad \text{удлинение крыла} \\
% \lambda_{\text{оп}} & = \frac{l_{\text{размах оп}}^2}{S_{\text{оп}} + b_{\text{кр}}^2} \quad \text{удлинение оперения}
% \end{align*}

% \subsection*{Координаты центров давления}

% \begin{align*}
% x_b & = 1.262 \; \text{м} \quad \text{координата центра давления крыла} \\
% x_{b_{\text{оп}}} & = 0.356 \; \text{м} \quad \text{координата центра давления оперения}
% \end{align*}

% \subsection*{Параметры пограничного слоя}

% \begin{align*}
% L_1 & = x_b + \frac{b_{\text{кр}}}{2} \; \text{м} \quad \text{длина передней части корпуса для крыла} \\
% L_{1_{\text{оп}}} & = x_{b_{\text{оп}}} + \frac{b_{\text{оп}}}{2} \; \text{м} \quad \text{длина передней части корпуса для оперения} \\
% \bar{L}_1 & = \frac{L_1}{D} \quad \text{относительная длина передней части (крыло)} \\
% \bar{L}_{1_{\text{оп}}} & = \frac{L_{1_{\text{оп}}}}{D} \quad \text{относительная длина передней части (оперение)}
% \end{align*}

% \subsection*{Физические константы}

% \begin{align*}
% \nu & = 15.1 \times 10^{-6} \; \text{м}^2/\text{с} \quad \text{кинематическая вязкость воздуха} \\
% \eta_k & = 1.0 \quad \text{коэффициент полноты корпуса} \\
% \kappa_M & = 0.96 \quad \text{поправочный коэффициент для } M = 2
% \end{align*}

% \subsection*{Параметры для функций скоса потока}

% \begin{align*}
% \alpha_p & = 0.0, \quad \varphi_\alpha = 0.0, \quad \psi_I = 0.0, \quad \psi_{II} = 0.0 \quad \text{угловые параметры} \\
% l_{1c_{II}} & = 0.036 \; \text{м} \quad \text{размах одной консоли} \\
% \zeta_{II} & = 0.0 \quad \text{обратное сужение} \\
% b_{b_{II}} & = b_{\text{кр}} \quad \text{длина бортовой хорды} \\
% \chi_{0_{II}} & = 0.0 \; \text{рад} \quad \text{угол стреловидности} \\
% L_{vI_{bII}} & = 0.882 \; \text{м} \quad \text{расстояние до точки схода вихря}
% \end{align*}

% \subsection*{Дополнительные геометрические параметры}

% \begin{align*}
% \lambda_{\text{нос}} & = 0.5 \quad \text{удлинение носовой части} \\
% \lambda_{\text{цил}} & = 20.1 \quad \text{удлинение цилиндрической части} \\
% L_{xv_k} & = 0.086 \; \text{м} \quad \text{длина задней части для крыла} \\
% L_{xv_{\text{оп}}} & = 1.068 \; \text{м} \quad \text{длина задней части для оперения}
% \end{align*}

% \subsection*{Вспомогательные коэффициенты}

% \begin{align*}
% Fi_1 & = 5.0 \\
% Fi_2 & = 0.4951 \\
% Fi_{2_{\text{оп}}} & = 5.0
% \end{align*}


\section{АДХ несущих поверхностей}


\section{АДХ управляющих поверхностей}

\section{АДХ всего ЛА}

\newpage
\begin{thebibliography}{9}
\bibitem{} \textbf{Nowell J. B. Jr.} Missile Total and Subsection Weight and Size. June 1992.
\bibitem{} \textbf{NAVY TRAINING SYSTEM PLAN} AIM-120 ADVANCED MEDIUM RANGE
 AIR-TO-AIR MISSILE. June 1998.

\end{thebibliography}

\end{document}